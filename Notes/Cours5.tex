\documentclass[a4paper,12pt]{article}

\usepackage[utf8]{inputenc}
\usepackage[T1]{fontenc}
\usepackage[french]{babel}
\usepackage{setspace}
\usepackage[top=2cm, left=2cm]{geometry}
\usepackage{graphicx}
\usepackage{eurosym}
\usepackage{palatino, eulervm} % Change font
\usepackage{hyperref}
\usepackage{mathtools, bm}
\usepackage{amssymb, bm}
\usepackage{color}
\usepackage{tcolorbox}
\usepackage{xcolor}
\usepackage{tikz}
\usetikzlibrary{arrows,automata}

\usepackage{fancyhdr}
\pagestyle{fancy}
% Head
\fancyhead[L]{}
\fancyhead[C]{\includegraphics[scale=0.4]{logo.jpg}}
\fancyhead[R]{}
% Footer
\fancyfoot[C]{}
% Line
\renewcommand{\headrulewidth}{0pt}

%opening
\title{INFO-F408: Computability \& complexity}
\date{16 Octobre, 2017}
\author{Rémy Detobel}

\begin{document}

\maketitle
\newpage

\section{Théorème de Rice}
  Pour $P$ n'importe quelle propriété ``non triviale'' d'un langage d'une machine de Turing:
  \begin{align*}
    \exists M_1 &\text{ tel que } L(M_1) \in P\\
    \wedge \exists M_2 &\text{ tel que } L(M_2) \notin P
  \end{align*}

  \begin{itemize}
   \item Problème de décision ($\equiv$ Langages)
   \item DFA (machine de turing déterministe)
   \item Non déterministes
   \item Power set construction: DFA (équivalent à NFA)
   \item Langage régulier (``Regular languages''), est l'ensemble des langages reconnus par un automate déterministe
   \itemsep1em
   \item Machines de Turing / Church-T thesis
   \itemsep0em
   \item Multitape Turing Machine / Machine de Turing non déterministe
   \item Décidabilité VS reconnaissable (``recognizable``) $\Leftrightarrow \exists$ un énumérateur (''enumerator``)
  \end{itemize}

  Un langage A est:
  \begin{align*}
    \text{reconnaissable (''recognizable``) } &\Leftrightarrow \exists M: L(M) = A\\
    \text{décidable } &\Leftrightarrow \exists M: \text{ M est un ''decider`` } L(M) = A
  \end{align*}

  \begin{itemize}
    \item Cantor's Diagonalization argument:\\
      $\Rightarrow \exists$ langage L $\in$ Reconnaissable
    \item Problème de l'arrêt (Halting problem)\\
      $A_{TM} = \{\langle M, w\rangle | M \text{ est une machine de Turing qui accepte } w\}$\\
      Reconnaissable et pas décidable $\overline {A_{TM}}$ n'est pas reconnaissable.
    \item Théorème de Rice.
  \end{itemize}

\section{Un problème indécidable simple}
  Voir livre chapitre 5.2\\
  $$\left\{\left[\frac{ab}{abab}\right], \left[\frac{b}{a}\right], \left[\frac{aba}{b}\right], \left[\frac{aa}{a}\right]\right\}$$
  (Emil) Post correspondence problem\\
  Théorème 5.15 PCP est indécidable (Post correspondence problem)\\

  $$M, w \in \Sigma^* \rightarrow (Q, \Sigma, \Gamma, \delta, q_0, q_{accept}, q_{reject})$$
  Construit une instance P du problème PCP.\\
  P (''has a match``) à une correspondance $\Leftrightarrow$ M accepte w.

  \begin{enumerate}
    \item Les premiers dominos sont:
      \begin{align*}
	&\#\\
	&\# q_0 w_1 w_2 ... w_n \#
      \end{align*}
    \item Pour chaque $a, b \in \Gamma$ et $q, r \in Q$ où $q \neq q_{reject}$\\
      Si $\delta(q, a) = (r, b, R)$ on ajoute $\left[\frac{qa}{br}\right]$
    \item $a, b, c \in \Gamma; q, r \in Q$\\
      Si $\delta(q, a) = (r, b, L)$ on ajoute $\left[\frac{cqa}{rcb}\right]$
    \item Pour chaque $a \in \Gamma$, on ajoute $\left[\frac{a}{a}\right]$
    \item Ajouter: $\left[\frac{\#}{\#}\right]$ et $\left[\frac{\#}{\textvisiblespace \_ \#}\right]$
    \item $\left[\frac{a q_{accept}}{q_{accept}}\right] \forall a \in \Gamma$ et $\left[\frac{q_{accept} a}{q_{accept}}\right]$
    \item $\left[\frac{q_{accept} \# \#}{\#}\right]$
  \end{enumerate}

  \textbf{Hypothèse:} une correspondance doit commencer par le premier domino (Modified PCP)\\

  \textbf{Exemple:}
  \begin{align*}
    > &\#\\
    > &\# q_0 0 1 0 0 \#\\
    &\delta(q_0, 0) = (q_7, 2, R)\\
    &\Big[\frac{q_0 0}{2 q_7}\Big]\\
    &\\
    > &\# q_0 0\\
    > &\# q_0 0 1 0 0 \# 2 q_7\\
    &\\
    > &\# q_0 0 1 0 0\\
    > &\# q_0 0 1 0 0 \# 2 q_7 1 0 0 \\
    &\\
    > &\# q_0 0 1 0 0 \#\\
    > &\# q_0 0 1 0 0 \# 2 q_7 1 0 0 \#\\
    &\delta(q_7, 1) = (q_1, 3, L)\\
    c= 2\\
    &\Big[\frac{2 q_7 1}{q_1 2 3}\Big]\\
    &\\
    > &\# q_0 0 1 0 0 \# 2 q_7 1 0 0 \#\\
    > &\# q_0 0 1 0 0 \# 2 q_7 1 0 0 \# q_1 2 3 0 0 \# \\
    &\\
  \end{align*}

  Comment supprimer la condition de M PCP ?  Posons:
  \begin{align*}
    \circledast w &= * u_1 * u_2 * u_3 ... * u_n\\
    u \circledast &= u_1 * u_2 * ... u_n *\\
    \circledast u \circledast &= * u_1 * u_2 * u_3 ... * u_n *
  \end{align*}
  On remplace donc:
  $$\left\{ \left[\frac{t_1}{b_1}\right] , \left[\frac{t_2}{b_2}\right], \ldots, \left[\frac{t_k}{b_k}\right] \right\}$$
  Par:
  $$\left\{ \left[\frac{\circledast t_1}{\circledast b_1 \circledast}\right], \left[\frac{\circledast t_2}{b_2 \circledast}\right], \ldots,
	\left[\frac{\circledast t_k}{b_k \circledast}\right] \right\}$$
  Avec cela ils doivent donc être en premier.  Mais on doit en plus rajouter~:
  $$\left[ \frac{\circledast \Diamond}{\Diamond} \right].$$


\section{Reducibility (Mapping/Many-One)}
  \textbf{Définition:} $f : \Sigma^* \rightarrow \Sigma^*$ est exécutable (computable) si et seulement s'il existe une machine de Turing $M$ telle que pour n'importe quel entrée w, M s'arrête avec seulement $f(w)$ sur son stack/ruban (tape)\\

  \textbf{Définition:} Un langage A est réductible dans un langage B, si il existe une fonction f exécutable (''computable``) telle que
  $$\forall w \in \Sigma^* : f(w) \in B \Leftrightarrow w \in A.$$
  On note cela~: $A \leq_M B$\\

  \textbf{Théorème:} Si $A \leq_M B$ et $B$ est décidable, alors $A$ est décidable.\\

  \textbf{Théorème:} Si $A \leq_M B$ et $B$ est reconnaissable (''recognizable''), alors $A$ est reconnaissable.
\end{document}
