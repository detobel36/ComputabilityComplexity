\documentclass[a4paper,12pt]{article}

\usepackage[utf8]{inputenc}
\usepackage[T1]{fontenc}
\usepackage[french]{babel}
\usepackage{setspace}
\usepackage[top=2cm, left=2cm]{geometry}
\usepackage{graphicx}
\usepackage{eurosym}
\usepackage{palatino, eulervm} % Change font
\usepackage{hyperref}
\usepackage{mathtools, bm}
\usepackage{amssymb, bm}
\usepackage{color}
\usepackage{tcolorbox}
\usepackage{xcolor}
\usepackage{tikz}
\usetikzlibrary{arrows,automata}

\usepackage{fancyhdr}
\pagestyle{fancy}
% Head
\fancyhead[L]{}
\fancyhead[C]{\includegraphics[scale=0.4]{../logo.jpg}}
\fancyhead[R]{}
% Footer
\fancyfoot[C]{}
% Line
\renewcommand{\headrulewidth}{0pt}

%opening
\title{INFO-F408: Computability \& complexity}
\date{16 Octobre, 2017}
\author{Rémy Detobel}

\begin{document}

\maketitle
\newpage

\section{Théorème de Rice}
  Pour $P$ n'importe quel propriété ``non trivial'' d'un langage d'une machine de turing:
  \begin{align*}
    \exists M_1 &\text{ tel que } L(M_1) \in P\\
    \wedge \exists M_2 &\text{ tel que } L(M_2) \notin P
  \end{align*}

  \begin{itemize}
   \item Problème de décision ($\equiv$ Langages)
   \item DFA (machine de turing déterministe)
   \item Non déterministes
   \item Power set construction: DFA (équivalent à NFA)
   \item Langage régulier (``Regular languages''), est l'ensemble des langages reconnu par un automate définit
   \itemsep1em
   \item Machines de turing / Church-T thesis
   \itemsep0em
   \item Multitape Turing Machine / Machine de turing non déterministe
   \item Décidabilité VS reconnaissable (``recognizability``) $\Leftrightarrow \exists$ un énumérateur (''enumerator``)
  \end{itemize}
  
  Un langage A est:
  \begin{align*}
    \text{reconnaissable (''recognizabile``) } &\Leftrightarrow \exists M: L(M) = A\\
    \text{décisable } &\Leftrightarrow \exists M: \text{ M est un ''decider`` } L(M) = A
  \end{align*}
  
  \begin{itemize}
    \item Cantor's Diagonalization argument:\\
      $\Rightarrow \exists$ langage L $\in$ Reconnaissable
    \item Problème de l'arret (Halting problem)\\
      $A_{TM} = \{<M, w> | M \text{ est une machine de turing qui accepte } w\}$\\
      Reconnaissable et pas décidable $\bar{A_{TM}}$ n'est pas reconnaissable.
    \item Théorème de Rice.
  \end{itemize}

\section{Un problème indécidable simple}
  Voir livre chapitre 5.2\\
  $$\Bigg\{\Big[\frac{ab}{abab}\Big], \Big[\frac{b}{a}\Big], \Big[\frac{aba}{b}\Big], \Big[\frac{aa}{a}\Big]\Bigg\}$$
  (Emil) Post correspondence problem\\
  Théorème 5.15 PCP est indécidable (Post correspondence problem)\\
  
  $$M, w \in \Sigma^* \rightarrow (Q, \Sigma, \Gamma, \delta, q_0, q_{accept}, q_{reject})$$
  Construit une instance P du problème PCP.\\
  P (''has a match``) à une correspondance $\Leftrightarrow$ M accepte w.
  
  \begin{enumerate}
    \item Les premiers dominos sont:
      \begin{align*}
	&\#\\
	&\# q_0 w_1 w_2 ... w_n \#
      \end{align*}
    \item Pour chaque $a, b \in \Gamma$ et $q, r \in Q$ où $q \neq q_{reject}$\\
      Si $\delta(q, a) = (r, b, R)$ on ajouter $\Big[\frac{qa}{br}\Big]$
    \item $a, b, c \in \Gamma; q, r \in Q$\\
      Si $\delta(q, a) = (r, b, L)$ on ajouter $\Big[\frac{cqa}{rcb}\Big]$
    \item Pour chaque $a \in \Gamma$, on ajouter $\Big[\frac{a}{a}\Big]$
    \item Ajouter: $\Big[\frac{\#}{\#}\Big]$ et $\Big[\frac{\#}{\textvisiblespace \_ \#}\Big]$
    \item $\Big[\frac{a q_{accept}}{q_{accept}}\Big] \forall a \in \Gamma$ et $\Big[\frac{q_{accept} a}{q_{accept}}\Big]$
    \item $\Big[\frac{q_{accept} \# \#}{\#}\Big]$
  \end{enumerate}
  
  \textbf{Assumption:} une correspondance doit commencer par le premier domino (Modified PCP)\\
  
  \textbf{Exemple:}
  \begin{align*}
    > &\#\\
    > &\# q_0 0 1 0 0 \#\\
    &\delta(q_0, 0) = (q_7, 2, R)\\
    &\Big[\frac{q_0 0}{2 q_7}\Big]\\
    &\\
    > &\# q_0 0\\
    > &\# q_0 0 1 0 0 \# 2 q_7\\
    &\\
    > &\# q_0 0 1 0 0\\
    > &\# q_0 0 1 0 0 \# 2 q_7 1 0 0 \\
    &\\
    > &\# q_0 0 1 0 0 \#\\
    > &\# q_0 0 1 0 0 \# 2 q_7 1 0 0 \#\\
    &\delta(q_7, 1) = (q_1, 3, L)\\
    c= 2\\
    &\Big[\frac{2 q_7 1}{q_1 2 3}\Big]\\
    &\\
    > &\# q_0 0 1 0 0 \# 2 q_7 1 0 0 \#\\
    > &\# q_0 0 1 0 0 \# 2 q_7 1 0 0 \# q_1 2 3 0 0 \# \\
    &\\
  \end{align*}
  
  Comment supprimer la condition de M PCP ?  Posons:
  \begin{align*}
    \circledast w &= * u_1 * u_2 * u_3 ... * u_n\\
    u \circledast &= u_1 * u_2 * ... u_n *\\
    \circledast u \circledast &= * u_1 * u_2 * u_3 ... * u_n *
  \end{align*}
  On remplace donc:
  $$\Bigg\{ \Big[\frac{t_1}{b_1}\Big] , \Big[\frac{t_2}{b_2}\Big], ... \Big[\frac{t_k}{b_k}\Big] \Bigg\}$$
  Par:
  $$\Bigg\{ \Big[\frac{\circledast t_1}{\circledast b_1 \circledast}\Big], \Big[\frac{\circledast t_2}{b_2 \circledast}\Big] ..., \Big[\frac{\circledast t_k}{b_k \circledast}\Big] \Bigg\}$$
  Avec cela ils doivent donc être en premier.  Mais on doit en plus rajouter donc:
  $$\Big[ \frac{\circledast \Diamond}{\Diamond} \Big]$$
  
  
\section{Reducibility (Mapping/Many-One)}
  \textbf{Définition:} $f : \Sigma^* \rightarrow \Sigma^*$ est exécutable (computable) si et seulement s'il existe une machine de turing M tel que pour n'importe quel entrée w, M s'arrête avec seulement $f(w)$ sur son stack/ruban (tape)\\
  
  \textbf{Définition:} Un langage A est réductible dans un langage B, si il existe une fonction f exécutable (''computable``) tel que
  $$\forall w \in \Sigma^* | f(w) \in B \Leftrightarrow w \in A$$
  Notons que $A \leq_m B$\\
  
  \textbf{Théorème:} Si $A \leq_m B$ et $B$ est décidable, alors $A$ est décidable.\\
  
  \textbf{Théorème:} Si $A \leq_m B$ et $B$ est reconnaissable (''recognizable''), alors $A$ est reconnaissable.
  
  
  
\end{document}