\documentclass[a4paper,12pt]{article}

\usepackage[utf8]{inputenc}
\usepackage[T1]{fontenc}
\usepackage[french]{babel}
\usepackage{setspace}
\usepackage[top=2cm, left=2cm]{geometry}
\usepackage{graphicx}
\usepackage{eurosym}
\usepackage{palatino, eulervm} % Change font
\usepackage{hyperref}
\usepackage{mathtools, bm}
\usepackage{amssymb, bm}
\usepackage{color}
\usepackage{tcolorbox}
\usepackage{xcolor}
\usepackage{tikz}
\usetikzlibrary{arrows,automata}

\usepackage{fancyhdr}
\pagestyle{fancy}
% Head
\fancyhead[L]{}
\fancyhead[C]{\includegraphics[scale=0.4]{../logo.jpg}}
\fancyhead[R]{}
% Footer
\fancyfoot[C]{}
% Line
\renewcommand{\headrulewidth}{0pt}

%opening
\title{INFO-F408: Computability \& complexity}
\date{9 Octobre, 2017}
\author{Rémy Detobel}

\begin{document}

\maketitle
\newpage

\section{Cantor's Diagonal}
  Corollary 4.18 dans le livre:\\
  Certain langages ne sont pas reconnaissable par une machine de Turing (Turing-recognizable).\\
  \textbf{Idée:}\\
  L'ensemble des machines des turing est dénombrable.\\

  \textbf{Exemple:}\\
  Tous les mots possibles dans l'alphabet: $\{0, 1\}$:\\

  \begin{table}[h]
    \centering
    \begin{tabular}{l|ccccccccc}
      & \multicolumn{1}{l}{$\varepsilon$} & \multicolumn{1}{l}{0} & \multicolumn{1}{l}{1} & \multicolumn{1}{l}{00} & \multicolumn{1}{l}{01} & \multicolumn{1}{l}{10} & \multicolumn{1}{l}{11} & \multicolumn{1}{l}{000} & \multicolumn{1}{l}{001} \\ \hline
      $M_1$ & \textbf{0}        & 1                     & 1                     & 0                      & 1                      & 1                      & 1                      & 0                       & 0  \\
      $M_2$ & 1                 & \textbf{1}            & 1                     & 1                      & 0                      & 1                      & 0                      & 1                       & 1  \\
      $M_3$ & 0                 & 1                     & \textbf{0}            & 1                      & 1                      & 0                      & 1                      & 1                       & 1  \\
      $M_4$ &                   &                       &                       &                        &                        &                        &                        &                         &    \\
      $M_5$ &                   &                       &                       &                        &                        &                        &                        &                         &
    \end{tabular}
  \end{table}
  On construit donc ici un table qui définit un langage.

\section{Halting Problem (problème de l'arrêt)}
  $$A_{TM} = \{ \langle M, w \rangle | M \text{ est une machine de Turing qui accepte } w\}$$
  $M$ = Une machine de Turing\\
  $w$ = un mot en entrée (input)\\
  Voir le livre, théorème 4.11: le problème $A_{TM}$ est indécidable.\\

  \subsection{Preuves}
    Par contradiction: supposons que $A_{TM}$ est décidable. Cela signifie qu'il existe une machine $H(\langle M,w\rangle) = \{\text{Accepte si M accepte w et rejette dans les autres cas}\}$\\
    On définit ensuite une machine $D$ telle que pour l'entrée $M (\text{une machine de turing})$:
    \begin{enumerate}
      \item Exécuter H sur l'entrée $\langle M, \langle M\rangle\rangle$
      \item On renvoie l'inverse de H: accepte si $M$ n'accepte pas $\langle M\rangle$ et rejette si $M$ accepte $\langle M\rangle$.
    \end{enumerate}
    Enfin, on exécute $D$ sur lui-même: $D(\langle D\rangle)$.\\
    Cela signifie que $D$ s'accepte uniquement s'il ne s'accepte pas, ce qui est une contradiction, et donc une telle machine $M$ ne peut exister.
  \\
  Pour un langage $A$, on définit $\bar{A}$ comme étant son complément: $\bar{A} = \{w | w \notin A\}$\\
  Supposons que $A$ et $\bar{A}$ sont (``recognizable'') reconnaissables.  Donc $A$ et $\bar{A}$ sont aussi décidables.\\
  \textbf{Preuve}: Posons $M$ et $M'$ reconnaissant respectivement $A$ et $\bar{A}$.  Construisons un ``décideur'' $D$ pour $A$ en exécutant $M$ et $M'$ en ``parallèle'' (en alternant étape par étape sur $M$ et sur $M'$).\\

  Posons maintenant $A_{TM}$ comme étant indécidable.  Est-il pour autant reconnaissable (``recognizable'') ?\\
  $A_{TM}$ est reconnaissable (preuve: simuler $M$ sur $w$).\\
  On peut également écrire:
  \begin{align*}
    A \text{ est décidable} \Leftrightarrow A \text{ et } \bar{A} \text{ sont reconnaissables}
  \end{align*}
  $\Rightarrow \overline {A_{TM}}$ n'est pas reconnaissable.\\
  $$\overline {A_{TM}} = \{\langle M, w\rangle | w \text{ n'est pas accepté par } M\}$$

\section{Reductibility (Réduction)}
  Pour une machine de Turing $M$, $L(M)$ décrit le langage reconnu par $M$.
  $$L(M) = \{w | M \text{ accepte w} \} \subset \Sigma^*$$
  $$REGULAR_{TM} = \{ \langle M\rangle | M \text{ est une machine de Turing et } L(M) \text{ est régulier}\}$$
  Rappel d'un langage régulier: qui peut être reconnu par un automate fini. Ce langage est indécidable.\\
  Voir \textbf{livre}, théorème 5.3\\
  \textbf{Preuve}:\\
    Définissons une Machine de Turing $M_2$ comme une fonction de $M$ (une machine de Turing) et $w$ ($\in \Sigma^*$)\\
    $M_2 = $ pour une certaine entrée x:
    \begin{enumerate}
      \item si x a la forme $0^{n}1^{n}$, on accepte
      \item sinon, on exécute $M$ sur $w$ et on accepte si $M$ accepte $w$.
    \end{enumerate}
    $M_2$ n'est pas spécialement un ``décideur''.  Cela va déprendre de $M$.  Notons également que $M_2$ est une ``fonction'' de $M$ et $w$.

  \textbf{Quel est le langage de $L(M_2)$ ?}\\
  \begin{enumerate}
    \item Si $M$ accepte $w$: alors $L(M_2) = \Sigma^*$ accepte tout.\\
      $\Rightarrow$ régulier
    \item Si $M$ n'accepte pas $w$, $L(M_2) = \{0^n 1^n | n \geq 0\}$\\
      $\Rightarrow$ pas régulier
  \end{enumerate}
  On a donc réussi à réduire le problème de l'arrêt à ce problème.\\
  Par contradiction, supposons que $REGULAR_{TM}$ est décidable et qu'il existe R, un ``décideur'' pour $REGULAR_{TM}$.\\
  Soit la machine de Turing $S$ telle que $S = $ en entrée $\langle M, w\rangle$~:
  \begin{enumerate}
    \item Construire $M_2$ pour $M$ et $w$
    \item Exécuter $R$ sur $M_2$ et on accepte si et seulement si $R$ accepte.
  \end{enumerate}
  En faisant cela, on montre que $S$ décide $A_{TM} \rightarrow$ contradiction

\section{Rice's Theorem}
  Chapitre 5 exercice 28\\
  Posons $P$ comme étant n'importe quelle propriété de langage NON-TRIVIAL (NONTRIVIAL).\\
  \textbf{Théorème}: Déterminer si le langage d'une machine de Turing a comme propriété $P$, est indécidable.\\
  $$\{\langle M\rangle | M \text{ est une machine de Turing et } L(M) \text{ a une propriété } P\}$$
  Une propriété triviale est une propriété sans importance, par exemple une propriété est triviale si tous les langages ou aucun langage ne l'a.\\

  \subsection{Démonstration}
    Par contradiction:\\
    Posons $R_P$ un ``décideur'' pour $L_P$.
    \begin{itemize}
      \item Posons $T_{\varnothing}$ comme une machine de Turing qui rejette toutes les possibilités ($L(T_{\varnothing}) = \varnothing$),
			supposons que $\varnothing \notin P$ (sans perte de généralité).
      \item Posons $T$ comme une machine de turing tel que $L(T) \in P$.\\
    \end{itemize}
    Prenons, $M$, $w$, construit tel que $M_w$: pour l'entrée x:
    \begin{enumerate}
      \item Simuler $M$ sur $w$.  Si c'est accepté, aller en étape 2.  Si c'est rejeté, on rejette.
      \item Simuler $T$ sur x, si c'est accepté on accepte, sinon on rejette.
    \end{enumerate}
    $M$ accepte $w$ est équivalent à dire que $L(M_w) \in P$.\\
    Maintenant nous pouvons donc créer un ``décideur'' pour $A_{TM}$ tel que:\\
    $S = $ pour une entrée $M$, $w$:
    \begin{enumerate}
      \item Construire $M_w$
      \item Exécuter $R_p$ sur $M_w$ et donner la même réponse.
    \end{enumerate}
    On a donc un ``décideur'' pour le problème de l'arrêt.  Ce qui n'est pas possible.  On a donc une contradiction.
\end{document}
