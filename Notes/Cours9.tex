\documentclass[a4paper,12pt]{article}

\usepackage[utf8]{inputenc}
\usepackage[T1]{fontenc}
\usepackage[french]{babel}
\usepackage{setspace}
\usepackage[top=2cm, left=2cm]{geometry}
\usepackage{graphicx}
\usepackage{eurosym}
\usepackage{palatino, eulervm} % Change font
\usepackage{hyperref}
\usepackage{mathtools, bm}
\usepackage{amssymb, bm}
\usepackage{multirow}
\usepackage{color}
\usepackage{tcolorbox}
\usepackage{xcolor}
\usepackage{tikz}
\usetikzlibrary{arrows,automata}

\usepackage{fancyhdr}
\pagestyle{fancy}
% Head
\fancyhead[L]{}
\fancyhead[C]{\includegraphics[scale=0.4]{../logo.jpg}}
\fancyhead[R]{}
% Footer
\fancyfoot[C]{}
% Line
\renewcommand{\headrulewidth}{0pt}

%opening
\title{INFO-F408: Computability \& complexity}
\date{27 novembre, 2017}
\author{Rémy Detobel}

\newtcolorbox{defBox}{%
colback=gray!5!white,colframe=black!75!white,fonttitle=\bfseries,
title=Définition}

\newtcolorbox{theoremeBox}{%
colback=blue!5!white,colframe=blue!40!white,fonttitle=\bfseries,
title=Théorème}

\newtcolorbox{importantBox}{%
colback=red!5!white,colframe=red!65!white,fonttitle=\bfseries,
title=Important}


\begin{document}

\maketitle
\newpage

\section{NP-Complete problems}
  \begin{itemize}
    \item 3-SAT
    \item CLIQUE
    \item INDEPENDENT SET (complément de CLIQUE) $\equiv$ CLIQUE in $\overline{G}$
    \item VERTEX COVER
    \item HAMILTONIAN PATH
    \item SUBSET SUM
  \end{itemize}
  
  \subsection{K-Independant Set in G}
    $\equiv$ CLIQUE in $\overline{G}$\\
    $\equiv (n-K)$ VERTEX COVER in G 
    
    \begin{center}
      \begin{tikzpicture}
	\node[draw, circle, fill=gray] (1) at (0,0) {};
	\node[draw, circle] (2) at (0,-1) {};
	\node[draw, circle] (3) at (-1,1) {};
	\node[draw, circle] (4) at (1,1) {};
	\node[draw, circle, fill=gray] (5) at (0,2) {};
	\node[draw, circle, fill=gray] (6) at (2,1) {};
	\node[draw, circle] (7) at (3,1) {};
	\node[draw, circle] (8) at (2,2) {};
	\node[draw, circle] (9) at (2,0) {};
	
	\node[draw, circle, fill=gray] (legende) at (0,-3) {};
	\node[] at (2, -3) {Vertex cover};
	
	\draw (1) -- (2);
	\draw (1) -- (3);
	\draw (4) -- (5);
	\draw (3) -- (5);
	\draw (1) -- (4);
	\draw (6) -- (4);
	\draw (6) -- (7);
	\draw (6) -- (8);
	\draw (6) -- (9);
      \end{tikzpicture}
    \end{center}
    
    \begin{tcolorbox}
      Lemme:\\
      Pour $G = (V, E)$\\
      Si $S \leq V$ est un vertex cover, alors $V_{S}$ est un INDEPENDANT SET
    \end{tcolorbox}
  
  \subsection{SUBSET SUM}
    \begin{align*}
      = &\Big\{ <S, t> | S = \{s_1, s_2, ... s_k\}\\
      &\exists \{y_1, ... y_l\} \subseteq S\\
      &\text{tel que} \sum\limits_{s \in S} s = t \Big\}
    \end{align*}
  
    \begin{theoremeBox}
      SUBSET-SUM est NP-Complet
    \end{theoremeBox}
    \textbf{Preuve:}\\
    Réduction depuis 3 SAT où $x_1 ... x_v$ sont des variables et où $c_1 .. c_n$ sont des clauses (conditions).
    
    \begin{table}[h]
      \centering
      \begin{tabular}{c|c|c|c|c|c|c|c|c|c|c|c|c|}
	\cline{3-13}
	\multicolumn{2}{c|}{} & $1$ & $2$ & $3$ & $...$ & $v$ &  & $1$ & $2$ & $3$ & $...$ & $m$ \\ \hline
	\multicolumn{1}{|c|}{} & $y_1$ & 1 & 0 & 0 & 0 & 0 &  &  & {\color[HTML]{32CB00} \textbf{1}} & 1 &  &  \\ \cline{2-13} 
	\multicolumn{1}{|c|}{\multirow{-2}{*}{$x_1$}} & $z_1$ & 1 & 0 & 0 & 0 & 0 &  &  &  &  &  &  \\ \hline
	\multicolumn{1}{|c|}{} & $y_2$ & 0 & 1 & 0 & 0 & 0 &  &  & 1 & 0 &  &  \\ \cline{2-13} 
	\multicolumn{1}{|c|}{\multirow{-2}{*}{$x_2$}} & $z_2$ & 0 & 1 & 0 & 0 & 0 &  &  &  & {\color[HTML]{FD6864} 1} &  &  \\ \hline
	\multicolumn{1}{|c|}{} & $y_3$ & 0 & 0 & 1 & 0 & 0 &  &  & 1 &  &  &  \\ \cline{2-13} 
	\multicolumn{1}{|c|}{\multirow{-2}{*}{$x_3$}} & $z_3$ & 0 & 0 & 1 & 0 & 0 &  &  &  &  &  &  \\ \hline
	\multicolumn{1}{|c|}{} & $g_1$ & 0 & 0 & 0 & 0 & 0 &  & 0 & 0 & 0 & * & 0 \\ \cline{2-13} 
	\multicolumn{1}{|c|}{\multirow{-2}{*}{*}} & $h_i$ & 0 & 0 & 0 & 0 & 0 &  & 0 & 0 & 0 & * & 0 \\ \hline
	\multicolumn{2}{|c|}{} &  &  &  &  &  &  &  &  &  &  &  \\ \hline
	\multicolumn{2}{|c|}{$t$} & 1 & 1 & 1 & 1 & 1 &  & 3 & 3 & 3 & 3 & 3 \\ \hline
      \end{tabular}
    \end{table}
    
    \begin{align*}
      C_2 &= (x_1 \vee x_2 \vee x_3)\\
      C_3 &= (x_1 \vee \overline{x_2} \vee x_4)
    \end{align*}
    \textcolor{red}{$C_3$} contient $\overline{X_2}$\\
    \textcolor{green}{$C_2$} contient $\overline{X_1}$\\
    
    *Pour chaque clause/colonne $c_i$, on inclut deux fois le nombre: $0...010...0$ (où le 1 est placé en index i).
    
    \begin{enumerate}
      \item Supposons qu'il existe une instance de SAT:\\
	  $\rightarrow$ construisons $s'$ comme ceci: $\forall i = 1 ... V$ si $x_i = T$ prenons $y_i$ dans $s'$, $z_i$ dans les autres cas.\\
	  Jusqu'a ce que cela satisfasse chaque clause où chaque 1, 2 ou 3 littéraux sont égaux à ``True''.\\
	  $\forall i = 1...m$:
	  \begin{itemize}
	    \item Si $c_i$ a 1 litéral à vrai qui inclut $g_i$ et $h_i$ dans $S'$
	    \item Si $c_i$ a 2 litéraux à vrai qui inclut $g_i$ seulement.
	  \end{itemize}
      \item Supposons qu'il existe $S' \subseteq S$ tel que $\sum S' = t$.  L’assignement est construit tel que:\\
	  $X_i = T \Leftrightarrow y_i \in S' (= F$ dans les autres cas$) \forall i = 1 ... V$\\
	  Et ceci \textbf{n'est pas un assignement valide} pour SAT
    \end{enumerate}
    
  
\section{Programmation dynamique}
  Algorithme pour SUBSET-SUM\\
  $x[i]$ est un nombre dans $S$ et $t$ est la somme a calculer.\\
  
  \textbf{Table T}\\
    $T[i][j] = True \Leftrightarrow$ il existe un sous-ensemble du premier nombre $i$ où la somme est $j$\\
    $i = 1 ... |S|$\\
    $j = 0 ... \sum S$\\
    
  \textbf{Algorithme}\\
    Initialisons $T$, appliqué pour $i = 1 ... |S|, j = 0 ... t$.\\
    $T[i][j] \leftarrow T[i-1][j] \vee T[i-1][j-x[i]]$\\
    $Return T[|S|][t]$\\
    n a la taille de l'entrée (par exemple: le nombre de bits).\\
    La complexité de l'algorithme DP pour le problème SUBSET SUM: $|S| \times t$\\
    
    $n \simeq \big( \sum\limits_{x_i \in S} \log_2 x_i \big) + \log_2 t$\\
    
    \subsection{Exemple}
      \begin{align*}
	S &= \{5\ 000\ 000\ 000, 939\ 000\ 000, 333\ 212\}\\
	t &= 5\ 939\ 333\ 212
      \end{align*}
      Complexité: $n \leq 4 \times 10 \times 4 = 160$\\
      $|S| \times t \geq 4 \times 5 \times 10^9$
  
  
\section{3-COLORING}
  Prenons un simple graphe non dirigé $G$, peut-on colorer ses points en 3 couleurs tel que deux point adjacent n'ont pas la même couleur.
  
  \begin{center}
      \begin{tikzpicture}
	\node[draw, circle, fill=green] (1) at (0,0) {};
	\node[draw, circle, fill=red] (2) at (-1,1) {};
	\node[draw, circle, fill=blue] (3) at (0,2) {};
	\node[draw, circle, fill=red] (4) at (1,1.5) {};
	\node[draw, circle, fill=green] (5) at (1,2.5) {};
	\node[draw, circle, fill=red] (6) at (2,2) {};
	\node[draw, circle, fill=blue] (7) at (3,3) {};
	\node[draw, circle, fill=green] (8) at (3,1) {};
	
	\draw (1) -- (2);
	\draw (1) -- (3);
	\draw (2) -- (3);
	\draw (4) -- (3);
	\draw (4) -- (1);
	\draw (4) -- (5);
	\draw (3) -- (5);
	\draw (6) -- (5);
	\draw (6) -- (7);
	\draw (6) -- (8);
	\draw (7) -- (8);
      \end{tikzpicture}
    \end{center}
    
  \subsection{Exercice 7.27}
    Prouver que 3-COLORING est NP-Complet
    
    \begin{center}
      \begin{tikzpicture}
	\node[draw, circle] (T) at (0,0) {T};
	\node[draw, circle] (F) at (1,1) {F};
	\node[draw, circle, fill=gray] (plein) at (2,0) {\_};
	\node[] at (1, -0.5) {Palette};
	
	\node[draw, circle] (x1) at (4,0) {};
	\node[draw, circle] (x2) at (5,0) {};
	\node[] at (4.5, -0.5) {Variable};
	
	\draw (T) -- (F);
	\draw (T) -- (plein);
	\draw (plein) -- (F);
	
	\draw (x1) -- (x2);
	\draw (plein) to[out=45,in=135] (x1);
	\draw (plein) to[out=90,in=135] (x2);
      \end{tikzpicture}
    \end{center}
    \begin{center}
      \begin{tikzpicture}
	\node[draw, circle] (1) at (2,1) {};
	\node[draw, circle] (2) at (1,0) {};
	\node[draw, circle] (3) at (3,0) {};
	
	\node[draw, circle] (4) at (3,-1) {};
	\node[draw, circle] (5) at (1,-1) {};
	
	\node[] at (-1, 0) {OR-Gadget};
	
	\draw (1) -- (2);
	\draw (1) -- (3);
	\draw (3) -- (2);
	
	\draw (3) -- (4);
	\draw (2) -- (5);
      \end{tikzpicture}
    \end{center}
    
    Il y a plusieurs manière de représenter le ``OR-GADGET'':\\
    \begin{minipage}{0.49\textwidth}
      \begin{tikzpicture}
	\node[draw, circle] (1) at (2,1) {\textcolor{green}{T}};
	\node[draw, circle] (2) at (1,0) {\textcolor{blue}{N}};
	\node[draw, circle] (3) at (3,0) {?};
	
	\node[draw, circle] (4) at (3,-1) {\textcolor{red}{F}};
	\node[draw, circle] (5) at (1,-1) {\textcolor{red}{F}};
	
	\node[] at (-0.5, 0) {Pas OK};
	
	\draw (1) -- (2);
	\draw (1) -- (3);
	\draw (3) -- (2);
	
	\draw (3) -- (4);
	\draw (2) -- (5);
      \end{tikzpicture}
    \end{minipage}
    \begin{minipage}{0.49\textwidth}
      \begin{tikzpicture}
	\node[draw, circle] (1) at (2,1) {\textcolor{green}{T}};
	\node[draw, circle] (2) at (1,0) {\textcolor{red}{F}};
	\node[draw, circle] (3) at (3,0) {\textcolor{blue}{N}};
	
	\node[draw, circle] (4) at (3,-1) {\textcolor{red}{F}};
	\node[draw, circle] (5) at (1,-1) {\textcolor{green}{T}};
	
	\node[] at (-0.5, 0) {OK};
	
	\draw (1) -- (2);
	\draw (1) -- (3);
	\draw (3) -- (2);
	
	\draw (3) -- (4);
	\draw (2) -- (5);
      \end{tikzpicture}
    \end{minipage}\\
    \begin{minipage}{0.49\textwidth}
      \begin{tikzpicture}
	\node[draw, circle] (1) at (2,1) {\textcolor{green}{T}};
	\node[draw, circle] (2) at (1,0) {\textcolor{blue}{N}};
	\node[draw, circle] (3) at (3,0) {\textcolor{red}{F}};
	
	\node[draw, circle] (4) at (3,-1) {\textcolor{green}{T}};
	\node[draw, circle] (5) at (1,-1) {\textcolor{green}{T}};
	
	\node[] at (-0.5, 0) {OK};
	
	\draw (1) -- (2);
	\draw (1) -- (3);
	\draw (3) -- (2);
	
	\draw (3) -- (4);
	\draw (2) -- (5);
      \end{tikzpicture}
    \end{minipage}
    \begin{minipage}{0.45\textwidth}
      \begin{tikzpicture}
	\node[draw, circle] (1) at (2,1) {\textcolor{red}{F}};
	\node[draw, circle] (2) at (1,0) {\textcolor{green}{T}};
	\node[draw, circle] (3) at (3,0) {\textcolor{blue}{N}};
	
	\node[draw, circle] (4) at (3,-1) {\textcolor{red}{F}};
	\node[draw, circle] (5) at (1,-1) {\textcolor{red}{F}};
	
	\node[] at (-0.5, 0) {OK};
	
	\draw (1) -- (2);
	\draw (1) -- (3);
	\draw (3) -- (2);
	
	\draw (3) -- (4);
	\draw (2) -- (5);
      \end{tikzpicture}
    \end{minipage}\\
    
    Réduction depuis 3 STA: 3 CNF formule $\phi$, variables: $x_1, ... x_l$:
    
    \textbf{Preuve}:
    \begin{center}
      \begin{tikzpicture}
	\node[draw, circle] (T) at (0,0) {\textcolor{green}{T}};
	\node[draw, circle] (F) at (2,0) {\textcolor{red}{F}};
	\node[draw, circle] (N) at (1,-1) {\textcolor{blue}{N}};
	
	\node[] at (-1, -3) {$x_1$};
	\node[draw, circle] (x1) at (0,-3) {};
	\node[draw, circle] (x11) at (2,-3) {};
	
	\node[] at (-1, -4) {$x_2$};
	\node[draw, circle] (x2) at (0,-4) {};
	\node[draw, circle] (x21) at (2,-4) {};
	
	\node[] at (1, -5) {...};
	
	\node[] at (-1, -6) {$x_l$};
	\node[draw, circle] (xl) at (0,-6) {};
	\node[draw, circle] (xl1) at (2,-6) {};
	
	
	\draw (T) -- (F);
	\draw (T) -- (N);
	\draw (N) -- (F);
	
	\draw (x1) -- (x11);
	\draw (x2) -- (x21);
	\draw (xl) -- (xl1);
	
	\draw (N) to[out=-160,in=90] (x1);
	\draw (N) to[out=-20,in=90] (x11);
	
	\draw (N) to[out=-175,in=140] (x2);
	\draw (N) to[out=-15,in=50] (x21);
	
	\draw (N) to[out=-175,in=160] (xl);
	\draw (N) to[out=-15,in=30] (xl1);
      \end{tikzpicture}
    \end{center}
    
    Par exemple: $(x_1 \vee \overline{x_2} \vee x_l$
    
    \begin{center}
      \begin{tikzpicture}
	\node[draw, circle] (T) at (0,0) {\textcolor{green}{T}};
	\node[draw, circle] (F) at (2,0) {\textcolor{red}{F}};
	\node[draw, circle] (N) at (1,-1) {\textcolor{blue}{N}};
	
	\node[] at (-1, -3) {$x_1$};
	\node[] at (3, -3) {$\overline{x_1}$};
	\node[draw, circle] (x1) at (0,-3) {};
	\node[draw, circle] (x11) at (2,-3) {};
	
	\node[] at (-1, -4) {$x_2$};
	\node[] at (3, -4) {$\overline{x_2}$};
	\node[draw, circle] (x2) at (0,-4) {};
	\node[draw, circle] (x21) at (2,-4) {};
	
	\node[] at (1, -5) {...};
	
	\node[] at (-1, -6) {$x_l$};
	\node[] at (3, -6) {$\overline{x_l}$};
	\node[draw, circle] (xl) at (0,-6) {};
	\node[draw, circle] (xl1) at (2,-6) {};
	
	
	\draw (T) -- (F);
	\draw (T) -- (N);
	\draw (N) -- (F);
	
	\draw (x1) -- (x11);
	\draw (x2) -- (x21);
	\draw (xl) -- (xl1);
	
	\draw (N) to[out=-160,in=90] (x1);
	\draw (N) to[out=-20,in=90] (x11);
	
	\draw (N) to[out=-175,in=140] (x2);
	\draw (N) to[out=-15,in=50] (x21);
	
	\draw (N) to[out=-175,in=160] (xl);
	\draw (N) to[out=-15,in=30] (xl1);
	
	\node[draw, circle] (y1) at (5, -2) {};
	\node[draw, circle] (y2) at (5.7,-2.5) {};
	\node[draw, circle] (y3) at (5,-3) {};
	
	\draw (y1) -- (y2);
	\draw (y3) -- (y2);
	\draw (y1) -- (y3);
	
	\node[draw, circle] (z1) at (6, -4) {};
	\node[draw, circle] (z2) at (6.7,-4.5) {};
	\node[draw, circle] (z3) at (6,-5) {};
	
	\draw (z1) -- (z2);
	\draw (z3) -- (z2);
	\draw (z1) -- (z3);
	
	\draw (x1) to[out=45,in=180] (y1);
	\draw (x21) to[out=30,in=180] (y3);
	
	\draw (y2) to[out=0,in=180] (z1);
	\draw (xl) to[out=30,in=180] (z3);
	\draw (z2) to[out=10,in=20] (N);
	\draw (z2) to[out=0,in=0] (F);
      \end{tikzpicture}
    \end{center}
    
    $\exists 3-coloring \Leftrightarrow y$ est satisfaisable.
    
    \begin{enumerate}
      \item Supposons que $\phi$ est satisfaisable: \\
	  $\exists$ un SAT associé a $f : \{x_1, ... x_n\} \rightarrow \{T, F\}$. Colorer la palette en \textcolor{red}{rouge} \textcolor{green}{vert} \textcolor{blue}{bleu} comme indiqué.\\
	  $\forall$ varaible $x_i$: colorier la variable gadget et:
	  \begin{tikzpicture}
	    \node[draw, circle, fill=green] (1) at (0, 0) {};
	    \node[draw, circle, fill=red] (2) at (1, 0) {};
	    \draw (1) -- (2);
	    
	    \node[draw, circle, fill=red] (3) at (0, -1) {};
	    \node[draw, circle, fill=green] (4) at (1, -1) {};
	    \draw (3) -- (4);
	    
	    \node[] at (0, -1.5) {$x_i$};
	    \node[] at (1, -1.5) {$\overline{x_i}$};
	  \end{tikzpicture}
	  Si $f(x_i) = T$ dans les autres cas\\
	  $\forall$ ``OR-GADGET'': utiliser la couleur A ou B pour le ``second'' ou gadgets et colorer A, B ou C pour le ``premier''.
      \item Supposons $\exists$ un 3-COLORING colorer en \textcolor{red}{rouge} (F) \textcolor{green}{vert} (T) \textcolor{blue}{bleu} (N).\\
	  Mettre $x_i$ a vrai si: 
	  \begin{tikzpicture}
	    \node[draw, circle, fill=green] (1) at (0, 0) {};
	    \node[draw, circle, fill=red] (2) at (1, 0) {};
	    \draw (1) -- (2);
	  \end{tikzpicture} et a faux si
	  \begin{tikzpicture}
	    \node[draw, circle, fill=red] (1) at (0, 0) {};
	    \node[draw, circle, fill=green] (2) at (1, 0) {};
	    \draw (1) -- (2);
	  \end{tikzpicture} 
    \end{enumerate}


  
\end{document}