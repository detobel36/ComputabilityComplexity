\documentclass[a4paper,12pt]{article}

\usepackage[utf8]{inputenc}
\usepackage[T1]{fontenc}
\usepackage[french]{babel}
\usepackage{setspace}
\usepackage[top=2cm, left=2cm]{geometry}
\usepackage{graphicx}
\usepackage{eurosym}
\usepackage{palatino, eulervm} % Change font
\usepackage{hyperref}
\usepackage{mathtools, bm}
\usepackage{amssymb, bm}
\usepackage{color}
\usepackage{tcolorbox}
\usepackage{xcolor}
\usepackage{tikz}
\usetikzlibrary{arrows,automata}

\usepackage{fancyhdr}
\pagestyle{fancy}
% Head
\fancyhead[L]{}
\fancyhead[C]{\includegraphics[scale=0.4]{../logo.jpg}}
\fancyhead[R]{}
% Footer
\fancyfoot[C]{}
% Line
\renewcommand{\headrulewidth}{0pt}

%opening
\title{INFO-F408: Computability \& complexity}
\date{2 Octobre, 2017}
\author{Rémy Detobel}

\begin{document}

\maketitle
\newpage

\section{Turing machine suite}
  \subsection{Non déterministe}
    $$\delta: Q \times \Gamma \rightarrow P ( Q \times \times \Gamma \times \{L, R\})$$
    \textbf{Voir livre}: théorème 3.16: Chaque NTM as un équivalent DTM.\\
    On va donc faire un parcoure de l'arbre en largeur (et non en profondeur).
  
  \subsection{Reconnaitre un langage de turing}
    \textbf{Voir théorème 3.21}:\\
    Un langage est ``turing-recognizable'' si et seulement si un ``enumerator'' l'énumère.

    \subsubsection{Démonstration}
      ($\Leftarrow$) Supposons qu'il existe un énumérateur ``E'':\\
      M = ``lorsque l'entrée est w''
      \begin{enumerate}
	\item Exécuter E: chaque fois que E écrit(/output) un string, on le compare avec w
	\item si c'est égal, on accepte.
      \end{enumerate}
    
      ($\Rightarrow$) Supposon qu'il existe une machine de turing qui reconnaisse le langage $L$.\\
      E = ``ignorer l'entrée''
      \begin{enumerate}
	\item Répéter pour $i = 1, 2, 3, ...$\\
	Exécuter M pour l'étape i, sur l'entrée $S_1, S_2, ... S_i$\\
	Si un exécution est acceptable, on affiche le $S_j$ correspondant.\\
	Au pire on fera ``i'' étapes pour afficher un mot, mais il pourra être affiché avant l'étape ``i''.
      \end{enumerate}
      
      \begin{table}[h]
	\centering
	\begin{tabular}{c|ccccc}
	step/input & $S_1$                  & $S_2$                  & $S_3$                  & $S_4$                  & $S_5$ \\ \hline
	1          & \multicolumn{1}{c|}{x} &                        &                        &                        &       \\ \cline{3-3}
	2          & x                      & \multicolumn{1}{c|}{x} &                        &                        &       \\ \cline{4-4}
	3          & x                      & x                      & \multicolumn{1}{c|}{x} &                        &       \\ \cline{5-5}
	4          & x                      & x                      & x                      & \multicolumn{1}{c|}{x} &       \\ \cline{6-6} 
	5          & x                      & x                      & x                      & x                      & x    
	\end{tabular}
      \end{table}
      
  \subsection{Langage régulier (regular languages)}
    Langage régulier = reconnaissable par un automate fini (FA: finite automaton)\\
    Langage décidable (= decidable = recursively) = décidable par une machine de turing.\\
    Langage reconnaissable (recognizable languages = recursively enumerable = RE) =
    \begin{itemize}
      \item recognized par une machine de turing
      \item a un enumerateur (``enumerator'')
    \end{itemize}
    
    Régulier < décidable < reconnaissable/recognizable
    
    
\section{The Church-Turing thesis}
  C'est une thèse, pas une preuve.\\
  $\Rightarrow$ La notion intuitive d'un algorithme est égal à un algorithme d'une machine de turing\\
  
  \subsection{Hilberts Problem}
    Est-ce qu'il existe un algorithme qui décide si un polynôme à une racine composée uniquement de nombre entier.\\
    \textbf{Exemple}:
    $$P(x) = x_1^2 + x_2 x_3^4 - 6 x_1 x_2^3 x_3 x_4^2 + 7 x_1$$
    Et on cherche donc des nombres entier $x_1, x_2, x_3, x_4$\\
    
    Il s'agit ici d'un problème ``recognazable'' (reconnaissable).  Car si il y a une solution, on pourra la voir.  Par contre, il n'est pas ``décidable'' parce que s'il n'y a pas de solution, il tournera à l'infini.\\
    
    L'indécidabilité de ce problème à été prouvé en 1970 par Matijasevic.
  
\section{Halting problème (problème de l'arrêt)}
  Point 4.2.\\
  Diagonalization (cantor)
  $f: A \rightarrow B$ est:\\
  ``un à un`` si tous les élément de A sont projeté de manière distincte sur des éléments de B.\\
  ''dans`` lorsque tous les éléments de A sont dans B, par exemple:\\
  $$\forall b \in B, \exists a \in A : f(a) = b$$
  ''one-to-one`` (un à un) ET ''onto'' (dans) = ``one-to-one correspondance''\\
  C'est équivalent à une bijection.
  
  Une ensemble A est ``countable'' (dénombrable) si il est finie OU t'il existe une correspondance un à un (``one-to-one correspondence'') entre A et $\mathbb{N}$ (ce qui est équivalent à dire qu'il a la même ``taille'' que $\mathbb{N}$).\\
  \textbf{Exemple}: est-ce que:
  \begin{itemize}
    \item les nombres paires sont dénombrable ?\\
      $\rightarrow$ Oui ($\mathbb{N} / 2$)
      
    \item les nombres rationnels ($\mathbb{Q}$) sont dénombrable ?\\
      $\rightarrow$ Oui (pour cela il faut juste mettre un ordre.  Pour se faire, on peut parcourir un tableau à double entrées représentant les numérateurs et dénominateurs. Il suffirait donc de simplement définir l'ordre de lecture qui logiquement se ferait plutôt en diagonal).
      
    \item $\mathbb{Z}$ est dénombrable ?\\
      $\rightarrow$ Oui (nombre négatif étant des paires, nombre positif étant des impaires. De cette manière on compte tous les nombres).
  \end{itemize}

  \subsection{Cantor's Diagonal}
    Théorème: $\mathbb{R}$ est indénombrable (``not countable'').\\
    Prouvons cela par contradiction:\\
    Supposons donc que $\mathbb{R}$ est dénombrable.  On a donc une liste qui fait correspondre tous les nombres naturels ($\mathbb{N}$) à un nombre présent dans $\mathbb{R}$).  On va donc prouver qu'il existe un $x \in [0, 1]$ qui n'est pas dans cette liste.
    \begin{table}[h]
      \centering
      \begin{tabular}{lr}
      \multicolumn{1}{l|}{1} & 0,\textbf{3}1415926535   \\
      \multicolumn{1}{l|}{2} & 1,0\textbf{0}000000000   \\
      \multicolumn{1}{l|}{3} & 22,12\textbf{3}12312312  \\
      \multicolumn{1}{l|}{4} & 323,010\textbf{1}0101010 \\
      \multicolumn{1}{l|}{5} & 4,1502\textbf{6}535010	\\
      \multicolumn{1}{l|}{6} & ...
      \end{tabular}
    \end{table}
    Pour construire le $x$, on va prendre le nom à la position i et l'incrémenter.  Ici $x$ vaut donc:
    $x = 0,41427...$  Donc, par construction, il ne peut pas être dans la liste.
  
  Prenons $\mathcal{L}$ comme étant l'ensemble des langages sur l'alphabet $\Sigma$\\
  Prouver que $\mathcal{L}$ est indénombrable (``uncountable'').
  
  
    
\end{document}