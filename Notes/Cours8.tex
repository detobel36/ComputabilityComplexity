\documentclass[a4paper,12pt]{article}

\usepackage[utf8]{inputenc}
\usepackage[T1]{fontenc}
\usepackage[french]{babel}
\usepackage{setspace}
\usepackage[top=2cm, left=2cm]{geometry}
\usepackage{graphicx}
\usepackage{eurosym}
\usepackage{palatino, eulervm} % Change font
\usepackage{hyperref}
\usepackage{mathtools, bm}
\usepackage{amssymb, bm}
\usepackage{color}
\usepackage{tcolorbox}
\usepackage{xcolor}
\usepackage{tikz}
\usetikzlibrary{arrows,automata}

\usepackage{fancyhdr}
\pagestyle{fancy}
% Head
\fancyhead[L]{}
\fancyhead[C]{\includegraphics[scale=0.4]{../logo.jpg}}
\fancyhead[R]{}
% Footer
\fancyfoot[C]{}
% Line
\renewcommand{\headrulewidth}{0pt}

%opening
\title{INFO-F408: Computability \& complexity}
\date{13 novembre, 2017}
\author{Rémy Detobel}

\newtcolorbox{defBox}{%
colback=gray!5!white,colframe=black!75!white,fonttitle=\bfseries,
title=Définition}

\newtcolorbox{theoremeBox}{%
colback=blue!5!white,colframe=blue!40!white,fonttitle=\bfseries,
title=Théorème}

\newtcolorbox{importantBox}{%
colback=red!5!white,colframe=red!65!white,fonttitle=\bfseries,
title=Important}


\begin{document}

\maketitle
\newpage

\section{Cook-Leving}
  \textbf{Rappel:} SAT est NP-Complet.\\
  Ce qui signifie que $A \in NP$ et $\exists$ une machine de Turing non déterministe $N$\\
  $\phi = \phi_{cell} \wedge \phi_{start} \wedge \phi_{accept} \wedge \phi_{move}$

  \subsection{Exemple}
    $$\delta(q_1, a) = \{(q_1, b, R)\}$$
    Qui signifie que si on lit $a$ sur le ruban, on écrit $b$ en $q_1$ et on se déplace vers la droite. On a donc ici une machine non déterministe, donc un ensemble de srotie,
		mais dans les faits cet ensemble n'a qu'un seul élément. Cette transition est donc déterministe.

    $$\delta(q_1, b) = \{(q_2, c, L), (q_2, a, R)\}$$
    Si on lit b sur le tape, on peut soit écrite $c$ sur $q_2$ et se déplacer vers la gauche ou aller à droite en écrivant $a$ sur $q_2$.

    \textbf{Tableaux}:\\
    \begin{center}
      \begin{minipage}[c]{.4\textwidth}
	\begin{tabular}{|c|c|c|}
	  \hline
	  a     & $q_1$ & b \\ \hline
	  $q_2$ & a     & c \\ \hline
	\end{tabular}\\
	Ce tableau est légal/valide
      \end{minipage}
      \begin{minipage}[c]{.4\textwidth}
	\begin{tabular}{|c|c|c|}
	  \hline
	  b & b & b \\ \hline
	  c & b & b \\ \hline
	\end{tabular}\\
	Ce tableau est légal/valide
      \end{minipage}

      \begin{minipage}[c]{.4\textwidth}
	\begin{tabular}{|c|c|c|}
	  \hline
	  \# & b & a \\ \hline
	  \# & b & a \\ \hline
	\end{tabular}\\
	Ce tableau est légal/valide
      \end{minipage}
      \begin{minipage}[c]{.4\textwidth}
	\begin{tabular}{|c|c|c|}
	  \hline
	  a & b & a \\ \hline
	  a & a & a \\ \hline
	\end{tabular}\\
	Ce tableau n'est \textbf{pas} valide
      \end{minipage}

      \begin{minipage}[c]{.4\textwidth}
	\begin{tabular}{|c|c|c|}
	  \hline
	  a     & $q_1$ & b \\ \hline
	  $q_2$ & a     & a \\ \hline
	\end{tabular}\\
	Ce tableau n'est \textbf{pas} valide
      \end{minipage}
      \begin{minipage}[c]{.4\textwidth}
	\begin{tabular}{|c|c|c|}
	  \hline
	  a & $q_1$ & a \\ \hline
	  a & b     & a \\ \hline
	\end{tabular}\\
	Ce tableau n'est \textbf{pas} valide
      \end{minipage}
    \end{center}

    $$* = \bigvee\limits_{a_1, a_2, ..., a_6\text{est une fenêtre légal}} \big(X_{i, j-1, a_1} \wedge X_{i, j, a_2} \wedge X_{i, j+1, a_3} \wedge X_{i+1, j, a_4} \wedge ... \wedge X_{i+1, j+1, a_6}\big)$$

    \begin{table}[h]
      \centering
      \begin{tabular}{cccc}
			      & j-1                         & j                          & j+1                        \\ \cline{2-4}
      \multicolumn{1}{c|}{i}   & \multicolumn{1}{c|}{$a_1$} & \multicolumn{1}{c|}{$a_2$} & \multicolumn{1}{c|}{$a_3$} \\ \cline{2-4}
      \multicolumn{1}{c|}{i+1} & \multicolumn{1}{c|}{$a_4$} & \multicolumn{1}{c|}{$a_5$} & \multicolumn{1}{c|}{$a_6$} \\ \cline{2-4}
      \end{tabular}
    \end{table}

    Si $\phi$ as une taille polynomiale, alors $\phi$ est satisfaisable, de manière équivalente, $\Leftrightarrow$ $w$ est accepté par N.


\section{3-SAT}
  Forme normale conjonctive se note ``CNF'' en anglais.  La formule de la FNC-4 s'écrit:
  $$\bigwedge\limits_{j} \big(l_{i_1, j} \vee l_{i_2, j} \vee l_{i_3, j} \big)$$
  On appelle donc ce qu'il y a dans le grand ``ET'' une clause, et chaque élément de cette close s'appelle un littéral et est soit une variable, soit la négation d'une variable.

  \subsection{3 SAT est NP-Complet}
    \begin{itemize}
      \item $\phi$ peut être écrit en FNC (seulement $\phi_{move}$ doit être transformé)
      \item Maintenant les clauses ne doivent pas avoir une taille de 3. Pourquoi faire ?
    \end{itemize}

    Supposons une clause: $(a_1 \vee a_2 \vee ... \vee a_l)$
    \begin{align*}
      &(a_1 \vee a_2 \vee Z_1) \wedge \\
      &(\overline{Z_1} \vee a_3 \vee Z_2) \wedge \\
      &(\overline{Z_2} \vee a_4 \vee Z_3) \wedge \\
      &... \wedge\\
      &(\overline{Z_{l-3}} \vee a_{l-1} \vee a_l)
    \end{align*}

    \textbf{Exemple}
    $a_4 = T, a_{i \neq 4} = F$
    \begin{align*}
      &(a_1 \vee a_2 \vee Z_1 \mathit{= T}) \wedge \\
      &(\overline{Z_1} \vee a_3 \vee Z_2 \mathit{= T}) \wedge \\
      &(\overline{Z_2} \vee a_4 \mathit{= T} \vee Z_3 \mathit{= F}) \wedge \\
      &(\overline{Z_3} \vee a_5\vee Z_4 \mathit{= F}) \wedge \\
      &(\overline{Z_4} ...) \\
    \end{align*}


\section{CLIQUE est NP-Complet}

  \subsection{Vertex Cover}
    $3\ SAT \leq_p Vertex\ Cover$

		\begin{figure}
		\centering
    \begin{tikzpicture}
      \node[draw, circle] (1_1) at (0,0) {$x_1$};
      \node[draw, circle] (1_2) at (2,0) {$x_2$};
      \node[draw, circle] (1_11) at (1,1) {$x_1$};

      \node[draw, circle] (2_2) at (4,0) {$\bar{x_2}$};
      \node[draw, circle] (2_21) at (6,0) {$\bar{x_2}$};
      \node[draw, circle] (2_1) at (5,1) {$\bar{x_1}$};

      \node[draw, circle] (3_2) at (8,0) {$x_2$};
      \node[draw, circle] (3_21) at (10,0) {$x_2$};
      \node[draw, circle] (3_1) at (9,1) {$x_1$};

      \draw (1_1) -- (1_2);
      \draw (1_1) -- (1_11);
      \draw (1_2) -- (1_11);

      \draw (2_1) -- (2_2);
      \draw (2_2) -- (2_21);
      \draw (2_21) -- (2_1);

      \draw (3_1) -- (3_2);
      \draw (3_2) -- (3_21);
      \draw (3_21) -- (3_1);

      \node[] at (12, 0.5) {CLAUSES};



      \node[draw, circle] (5_1) at (0,4) {$x_1$};
      \node[draw, circle] (5_11) at (2,4) {$\bar{x_1}$};

      \draw (5_1) -- (5_11);

      \node[draw, circle] (6_2) at (8,4) {$x_2$};
      \node[draw, circle] (6_21) at (10,4) {$\bar{x_2}$};

      \draw (6_2) -- (6_21);

      \node[] at (12, 4) {VARS};

      \draw[red] (1_1) to[out=90,in=-100] (5_1);
      \draw[red] (1_11) -- (5_1);
      \draw[red] (3_1) -- (5_1);

      \draw[red] (2_1) -- (5_11);

      \draw[red] (1_2) -- (6_2);
      \draw[red] (3_2) -- (6_2);
      \draw[red] (3_21) to[out=90,in=-60] (6_2);

      \draw[red] (2_2) to[out=90,in=-100] (6_21);
      \draw[red] (2_21) to[out=90,in=-100] (6_21);

    \end{tikzpicture}
		\end{figure}

    K-Vertex cover:\\
    \begin{align*}
      \phi &= (X_1 \vee x_1 \vee x_2) \wedge \\
      &(\bar{x_1} \vee \bar{x_2} \vee \bar{x_2}) \wedge\\
      &(x_1 \vee x_2 \vee x_2)
    \end{align*}
    Une solution est de mettre $x_1 = T$ et $x_2 = F$

\section{Hamiltonian Path}
  Hamiltonian VS Eulerian\\
  Euler: ``Bridge of Königsberg''

	\begin{figure}
	\centering
  \begin{tikzpicture}
    \node[draw, circle] (a) at (0,0) {a};
    \node[draw, circle] (b) at (0,-2) {b};
    \node[draw, circle] (c) at (2,0) {c};
    \node[draw, circle] (d) at (2,2) {d};
    \node[draw, circle] (e) at (4,2) {e};
    \node[draw, circle] (f) at (4,0) {f};
    \node[draw, circle] (g) at (2,-2) {g};
    \node[draw, circle] (h) at (4,-2) {h};
    \node[draw, circle] (i) at (2,-4) {i};

    \draw (a) --node[inner sep=0pt, anchor=east]{1} (b);
    \draw (b) --node[inner sep=0pt, anchor=south]{12} (g);
    \draw (g) --node[inner sep=0pt, anchor=west]{7} (c);
    \draw (f) --node[inner sep=0pt, anchor=north]{6} (c);
    \draw (f) --node[inner sep=0pt, anchor=east]{9} (h);
    \draw (g) --node[inner sep=0pt, anchor=north]{8} (h);
  \end{tikzpicture}
	\end{figure}

  $$HamPath = \{\langle G, S, t\rangle | G \text{ est un graphe dirigé avec un chemin Hamiltonien de S a t}\}.$$
  \begin{theoremeBox}
    Hamiltonian Path est NP-Complet
  \end{theoremeBox}
  Pour une formule 3-FNC $\phi$, on peut construire $G, s, t$ tels que $\phi$ est SAT $\Leftrightarrow$ G a un chemin Hamiltonien de S à t.

  \begin{minipage}{0.3\textwidth}
    \centering
    \begin{tikzpicture}
      \node[draw, circle] (s) at (1,10) {S};
      \node[draw, circle] (x1) at (0,8) {$x_1$};
      \node[draw, circle] (nx1) at (2,8) {$\bar{x_1}$};

      \node[draw, circle] (s2) at (1,6) {};
      \node[draw, circle] (x2) at (0,4) {$x_2$};
      \node[draw, circle] (nx2) at (2,4) {$\bar{x_2}$};

      \node[draw, circle] (sn) at (1,0) {};
      \node[draw, circle] (xn) at (0,-2) {$x_n$};
      \node[draw, circle] (nxn) at (2,-2) {$\bar{x_n}$};
      \node[draw, circle] (t) at (1,-4) {t};

      \draw[-> latex] (s) -- (x1);
      \draw[-> latex] (s) -- (nx1);

      \draw[-> latex] (x1) -- (s2);
      \draw[-> latex] (nx1) -- (s2);

      \draw[-> latex] (s2) -- (x2);
      \draw[-> latex] (s2) -- (nx2);

      \draw[-> latex] (sn) -- (xn);
      \draw[-> latex] (sn) -- (nxn);
      \draw[-> latex] (xn) -- (t);
      \draw[-> latex] (nxn) -- (t);

      \draw[-> latex, dashed] (x2) -- (0.9,2.3);
      \draw[-> latex, dashed] (nx2) -- (1.1,2.3);

      \draw[dashed] (1,2) -- (1,0.5);

    \end{tikzpicture}
  \end{minipage}
  Variable Gadgets pour chaque variable $x_i$

  \begin{tikzpicture}
    \foreach \x in {0,...,10}
      \node[draw, circle] (s\x) at (\x,0) {};
    \foreach \x in {0,...,9} {
      \draw[-> latex] (s\x) to[out=45,in=135] (\x+1-0.15, 0.1);
      \draw[-> latex] (s\x) to[out=-45,in=225] (\x+1-0.15, -0.1);
    }
  \end{tikzpicture}

  ``Hard'' direction: $\exists$ un chemin Hamiltonien $\Rightarrow \exists$ une assignation (valuation) au problème de SAT.

\end{document}
