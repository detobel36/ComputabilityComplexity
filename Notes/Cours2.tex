\documentclass[a4paper,12pt]{article}

\usepackage[utf8]{inputenc}
\usepackage[T1]{fontenc}
\usepackage[french]{babel}
\usepackage{setspace}
\usepackage[top=2cm, left=2cm]{geometry}
\usepackage{graphicx}
\usepackage{eurosym}
\usepackage{palatino} % Change font
\usepackage{hyperref}
\usepackage{mathtools, bm}
\usepackage{amssymb, bm}
\usepackage{color}
\usepackage{tcolorbox}
\usepackage{xcolor}
\usepackage{tikz}
\usetikzlibrary{arrows,automata}

\usepackage{fancyhdr}
\pagestyle{fancy}
% Head
\fancyhead[L]{}
\fancyhead[C]{\includegraphics[scale=0.4]{../logo.jpg}}
\fancyhead[R]{}
% Footer
\fancyfoot[C]{}
% Line
\renewcommand{\headrulewidth}{0pt}

%opening
\title{INFO-F408: Computability \& complexity}
\author{Rémy Detobel}

\begin{document}

\maketitle
\newpage

\section{Automates}
  \subsection{Non déterministe}
    \textbf{DFA}
    $$\delta : Q \times Z \rightarrow Q$$
    $Q \rightarrow$ liste des états\\
    $Z \rightarrow$ l'alphabet (liste de caractère)\\
    $\delta \rightarrow$ langage\\
    \\
    Pour les automates non déterministes, on pourra utiliser le symbole $\epsilon$ représentant un caractère vide.  Mais également avoir deux transition d'un état vers d'autres états où la transition serait la même...  On a donc quelque chose de non déterministe.\\
    Dans ce cas la, le second $Q$ sera donc une liste de liste d'état possible.\\
    $$\delta : Q \times Z_{\epsilon} \rightarrow \bm{P}(Q)$$
    Attention, il y a une différence entre: $P(Q)$ et $2^Q$.\\
    \textbf{Exemple:}\\
    \begin{align*}
      Q &= \{q_1, q_2, q_3\}\\
      P(Q) &= \{\{q_1\}, \emptyset, \{q_2\}, \{q_3\}, \{q_1, q_2\}, \{q_2, q_3\}, ...\}\\
      |P(Q)| &= 2^{|Q|}
    \end{align*}
    Voir livre: théorème 1.39 (page 55):\\
    Chaque NFA a un équivalent ou un équivalent en DFA (qui reconnait le même langage).\\
    Concernant le chapitre 1, nous n'avons pas tout vu. Il ne faut donc pas tout connaitre...
    
    \subsubsection{Transformer un NFA en DFA}
      \begin{align*}
	&\text{NFA} \\
	N &= (Q, \Sigma, \delta, q_0, F)\\
	&\text{DFA} \\
	M &= (Q', \Sigma, \delta', q_0', F')
      \end{align*}
      On va donc faire ça étape par étape:
      \begin{align*}
	Q' &= P(Q)\\
	\delta'(R, a) &= \cup_{r \in R} \delta(r, a)\\
	q_0' &= \{q_0\}\\
	F' &= \{R \in Q' | R \text{ contains an element of } F\}\\
	& \leftrightarrow R \cap F \neq \emptyset
      \end{align*}
      Ce changement à un inconvénient: il est exponentiel...  Mais dans le cas présent nous n'allons pas nous occuper de la taille des automates.
    
    
\section{The Church-Turing Thesis}
  Livre: Chapitre 3\\
  De manière intuitive, on peut dire qu'une machine de Turing à au moins une mémoire.  Qui sera appelée un ``TAPE'' (un ruban).  Sur ce ruban on met donc une tête de lecture (head) pour lire les éléments un à un.  On se déplace donc vers la gauche ou vers la droite.\\
  On peut ensuite faire des opérations comme remplacé l'information et se déplacé vers la droite.  Une grande différence (d'un point de vue conceptuel) avec les pc d'aujourd'hui réside dans le fait qu'il n'y ai pas d'accès ``random``/aléatoire à la mémoire.  Il faut faire plusieurs opérations.\\
  On considérera ici que le ruban est infini et un nombre d'état fini ainsi qu'une fonction de transition.
  $$\delta : Q \times \Gamma_1 \rightarrow Q \times \Gamma_2 \times \{L, R\}$$
  $Q \rightarrow$ liste des états\\
  $\Gamma_1 \rightarrow$ alphabet sur le ruban\\
  $\Gamma_2 \rightarrow$ lettre que l'on écrit sur le ruban\\
  $\{L, R\}$ gauche ou droite\\
  $$(Q, \Sigma, \Gamma, \delta, q_0, q_{\text{accepté}}, q_{\text{rej.}})$$
  $\Sigma \rightarrow$ alphabet présent sur le ruban\\
  $\Gamma \rightarrow$ alphabet pouvant se trouver sur le ruban (potentiellement écrit par la machine)\\
  $$\Sigma \subseteq \Gamma$$\\
  \textbf{Exemple 3.9}
  Décision:
  $$B = \{w \# w | w \in \{0, 1\}^*\}$$
  \textbf{Exemple}\\
  $00\#01 \in B \rightarrow$ Faux\\
  $010\#010 \in B \rightarrow$ Vrai\\
  \\
  \begin{center}
    \begin{figure}[h]
      \includegraphics[scale=0.5]{./Cours2_Ex3_9.jpg}
    \end{figure}
  \end{center}
  
  Lorsque l'on est en $q_1$ et que l'on voit un 0, on peut écrire: $\delta(q_1, 0) = (q_2, x, R)$\\
  \begin{align*}wifi
    \delta(q_2, 0) &= (q_2, 0, R)\\
    \delta(q_2, 1) &= (q_2, 1, R)
  \end{align*}
  Note: ''\textvisiblespace`` désigne le symbole ''blanc``, vide...
  
  \subsection{''Turing-Decidable``}
    Un langage $L$ est ''Turing-Decidable`` $\Leftrightarrow$ il existe une machine de turing qui s'arrête sur toutes les entrées, comme si L était la liste des états accepté.\\
    Une machine de turing s'arrête sur toutes les entrées est appelé ''Decider``.\
    
  \subsection{''Turing-Recognazable``}
    Un langage $L$ est ''Turing-Recognizable`` $\Leftrightarrow$ il existe une machine de turing tel que L est une liste d'état d'entré acceptable.  Parfois ici la machine ne s'arrête pas (et ces mots ne se donc pas dans le langage L).
    
  \subsection{Différence entre les deux}
    \textbf{Decidable}\\
    Si on prend l'entrée x.  Si il existe une machine de turing tel que:
    \begin{itemize}
      \item si $x \in L$:  cela s'arrête sur $q_{\text{accepté}}$
      \item si $x \notin L$:  cela s'arrête sur $q_{\text{rejeté}}$
    \end{itemize}

    \textbf{Reconnaitre}\\
    Si on prend l'entrée x.  Si il existe une machine de turing tel que:
    \begin{itemize}
      \item si $x \in L$:  cela s'arrête sur $q_{\text{accepté}}$
      \item si $x \notin L$:  cela s'arrête sur $q_{\text{rejeté}}$ OU il ne s'arrête jamais.
    \end{itemize}
    Par définition, les langages qui sont décidable, sont des langages que l'on peut ''reconnaitre``.
  
  \subsection{Variantes des machines de turing}
    Livre 3.2 (page 148)
  
    \subsubsection{Multitape Turing machine}
      Si l'on a 3 rubans (égal à K), on pourrait donc écrire de manière mathématique
      $$\delta : Q \times \Gamma^{k} \rightarrow Q \times \Gamma^K \times \{L, R, S\}^K$$
      Où $\{L, R, S\}$ correspond (respectivement) à gauche (left), droite (right), rester (stay).\\
      \textbf{Exemple: } $\delta(q, (a, b, c)) \rightarrow (q', (a, b, d), (L, R, S))$\\
      \textbf{Théorème 3.13:} Chaque multitape Turing machine a un équivalent (single-tape) en Turing Machine.
      
    \subsubsection{Simuler du Multitape sur une simple Turing Machine}
      L'idée est de ''simuler`` les ''k tapes`` sur une seule en se déplaçant. Pour ce faire, on augmente l'alphabet $\Gamma$ avec ''\#`` et le symbole $\dot{x}$ pour chaque $x \in \Gamma$ représentant les têtes.\\
      Ici on ne fait pas attention à la complexité, l'efficacité.  C'est évidemment plus lourd de tout mettre sur un ruban mais l'important c'est que ça soit possible.
      
    \subsubsection{Machine de Turing non déterministe}
      $$\delta : Q \times \Gamma \rightarrow P (Q \times \Gamma \times \{L, R\})$$
      On va donc avoir des ''branches`` d’exécution.  Quand l'entrée est acceptée, cela signifie qu'il existe une branche menant à $q_{\text{accepté}}$
    
  \subsection{Computation tree (arbre d'exécution)}
    \textbf{Théorème 3.16} Chaque NTP a un équivalent (D)TM (qui reconnait donc le même langage).\\
    Il y a plusieurs moyen de parcourir cet arbre: parcours en profondeur, parcourt par niveau, parcours en largeur.  C'est ce dernier algorithme qui sera privilégié.  Pour faire cela avec un TM, on va utiliser un 3-Tape (3 rubans) turing machine déterministe.\\
    On aura donc 3 rubans avec: l'entrée, la simulation et l'adresse.  En simulant cela, on va pouvoir voir si on arrive sur un statut accepté.
    
  
\end{document}