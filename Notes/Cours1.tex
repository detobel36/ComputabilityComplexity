\documentclass[a4paper,12pt]{article}

\usepackage[utf8]{inputenc}
\usepackage[T1]{fontenc}
\usepackage[french]{babel}
\usepackage{setspace}
\usepackage[top=2cm, left=2cm]{geometry}
\usepackage{graphicx}
\usepackage{eurosym}
\usepackage{palatino, eulervm} % Change font
\usepackage{hyperref}
\usepackage{color}
\usepackage{tcolorbox}
\usepackage{xcolor}
\usepackage{tikz}
\usetikzlibrary{arrows,automata}

\usepackage{fancyhdr}
\pagestyle{fancy}
% Head
\fancyhead[L]{}
\fancyhead[C]{\includegraphics[scale=0.4]{logo.jpg}}
\fancyhead[R]{}
% Footer
\fancyfoot[C]{}
% Line
\renewcommand{\headrulewidth}{0pt}

%opening
\title{INFO-F408: Computability \& complexity}
\author{Rémy Detobel}
\date{18 Septembre 2017}

\begin{document}

\maketitle
\newpage

\section{Introduction}
  C'est donc un cours de math, permettant de classer les problèmes, leur complexité, ... Il s'agit de la suite du cours d'``informatique fondamentale``.

  \subsection{Référence du cours}
    \begin{itemize}
      \item Introduction to the Theory of computation 2\up{nd} Ed.  Michael SIPSER.
      \item Hilbert (mathématicien), Gödel (il aurait prouvé que complètement transformer les mathématiques en ''programme`` n'est pas possible), Turing, Kleene, Church, Cook-Levin (NP et ses implications), Karp
    \end{itemize}

  \subsection{Définition}
    \begin{itemize}
      \item \textbf{What problems can be solved by a computer ?}\\
	  On doit définir plusieurs mots: ''problems``, ''solved`` et ''computer``.
      \item ''Efficace``: quel est réellement sa définition.  Que veut-on optimiser: le temps, l'énergie utilisé, la place, ressource naturelle (pour créer la machine par exemple).
      \item \textbf{How much resources are ''required`` to solve a problem ?}
      \item Parfois deux problèmes sont très différent mais en résolvant un des deux, on peut très facilement solutionner le problème en changeant peu l'algorithme.
      \item \textbf{Mathematical Treatment}\\
	Il y a un certaine formalisme à avoir pour pouvoir caractériser, classer, ... les problèmes.  Il y a différent niveau de langages.  Il ne faut pas se satisfaire de comprendre en général mais bien en profondeur.
      \item \textbf{Compilateurs, langages de programmation}
      \item \textbf{Optimisation: problème linéaire (LP), algorithme génétique, problème de satisfaisabilité, plus court chemin}
      \item \textbf{P vs NP (traveling problem (voyageur de commerce))}
    \end{itemize}

  \subsection{Connexion avec d'autres matières}
    Biologie, physique (ordinateur quantum (pas abordé dans ce cours)), ...

  \subsection{Quelques concepts de bases}
    Sets, logique booléen, séquences, relations, fonctions, graphes, ''strings``/chaînes de caractères, preuve par l'absurde (''proof by contradiction'').\\
    Référence: chapitre 0 du livre pour se rappeler.

  \subsection{Problème de décision}
    Problème où la réponse est ``oui'' ou ``non``.  Certains peuvent être résolus par des algorithmes et des ordinateurs.  Il faut donner des informations qui permettent de décider si la réponse est ''oui`` ou ''non''.  C'est là qu'est le problème principal des algorithmes.  On va donc utiliser seulement des problèmes dont la solution est déjà définie par l'entrée (e.g. y a-t-il un chemin entre deux points d'un graphe).\\
    \begin{itemize}
     \item well-defined
     \item self-contrained
    \end{itemize}
    Il y a une différence entre problème et une instance.  Le problème est plus général qu'une instance.  Une instance est un cas particulier d'un problème, dans lequel on a donc définit les entrées/inputs.  Chaque ``input'' possible est donc une instance...\\
    Problème: la liste des entrée pour lesquelles l'entrée est ``oui''.

    \subsubsection{Exemples}
      \textbf{Problème:} dire si un nombre est pair ou non.\\
      \textbf{Solution:} tous les nombres pairs (donc tous les nombres impairs sont faux).\\
      \\
      \textbf{Problème:} un graphe est-il connexe (donc est-il possible d'aller de n'importe quel point dans le graphe vers n'importe quel autre point).\\
      \textbf{Solution:} tous les graphes connexes (on peut les encoder sous forme de matrice (0 pas de lien, 1 si un lien)).

    \subsubsection{Input/Entrée}
      Les entrées/input doivent appartenir à un langages (qui peut être $\{0, 1\}$): une liste de mots/chaînes de caractères.  On considérera aussi que lorsque l'input n'est pas bon, la réponse est ``non''.  Pour le second exemple donc, si on ne reçoit pas un graphe en entrée, on répondra ``non''. Il faudra donc que l'algorithme vérifie d'abord l'entrée avant d'essayer de résoudre.

  \subsection{Problème de décision $\neq$ problème d'optimisation}
    Bisection/dichotomie\\
    Un problème de décision peut aider à savoir s'il est optimisé...\\
    Exemple (problème de décision): est-ce que le chemin coûte moins que la valeur ``d'' ?
    On prend ``d`` comme le milieu entre 0 et le maximum (nombre de point dans le graphe).\\
    \textbf{Transformé en problème de décision}: en général c'est possible en reformulant le problème.

\section{Automate fini}
  Cela permet de déjà commencer à répondre à la question ''qu'est-ce qu'un ordinateur``.
  5-uple: $(Q, \Sigma, \delta, q_0, F)$\\
  Référence: chapitre 1.1 du livre\\
  \\
  $Q$ = nombre fini d'état (finite set of states).\\
  $\Sigma$ = alphabet utilisé par le problème (finite set called the alphabet)\\
  $\delta : Q \times \Sigma \rightarrow Q$ (produit cartésien (cartesian product)) fonction de transition\\
  $q_0 \in Q$ c'est l'élément de départ (the start state)\\
  $F \subseteq Q$ liste des états finaux (set of accepting states)\\

  \subsection{Regular language}
    Définition: livre point 1.16\\
    A language is called a regular language if some finite automaton recognizes it. (= Regular Exepression (REGEX))\\
    \textit{Finite automaton M} defines a language $L(M)$.\\
    \textbf{Exemple}: $A = \{w \in \{0, 1\}* | w$ contains at least on 1 and an even number of 0 follow the last 1$\}$

\section{No determinism / Non déterministe}
  Référence: chapitre 1.2 du livre\\
  Les automates finis sont déterministes.\\
  $\epsilon$ = ''empty'' symbol
  
  \begin{figure}[h!]
    \centering
    \begin{tikzpicture}[->,>=stealth',shorten >=1pt,auto,node distance=2.8cm,
		      semithick]
      \tikzstyle{every state}=[fill=gray!30!white,draw=none,text=black]
      \tikzstyle{accepting}=[fill=gray!30!white,draw=black,text=black]

      \node[initial,state] (1) {$q_1$};
      \node[state] (2) [right of=1] {$q_2$};
      \node[state] (3) [right of=2] {$q_3$};
      \node[state,accepting] (4) [right of=3] {$q_4$};

      \path (1) edge [loop below] node {0, 1} (1)
	    (1) edge node {1} (2)
	    (2) edge node {0, $\epsilon$} (3)
	    (3) edge node {1} (4);
	    (4) edge [loop below] node {0, 1} (4);
    \end{tikzpicture}
  \end{figure}
  
  Dans un automate non déterministe le $\delta$ ne sera pas définit de la même manière.  Il sera écrit:
  $$\delta : Q \times \Sigma_{\epsilon} \rightarrow \mathcal P(Q)$$
  $\mathcal P(Q)$ liste d'états dans Q (set of states in Q)\\
  Les automates ``non déterministes'' reconnaissent exactement les mêmes langages que les automates ``déterministes''.  Cela dit, les automates non déterministes sont plus simple à écrire et à comprendre au premier abord (la preuve arrivera).\\
  Voir livre 1.39.\\
  Chaque NFA (automate non déterministe) à un équivalent en DFA (automate déterministe).

\end{document}
