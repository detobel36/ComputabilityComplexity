\documentclass[a4paper,12pt]{article}

\usepackage[utf8]{inputenc}
\usepackage[T1]{fontenc}
\usepackage[french]{babel}
\usepackage{setspace}
\usepackage[top=2cm, left=2cm]{geometry}
\usepackage{graphicx}
\usepackage{eurosym}
\usepackage{palatino} % Change font
\usepackage{hyperref}
\usepackage{mathtools, bm}
\usepackage{amssymb, bm}
\usepackage{color}
\usepackage{tcolorbox}
\usepackage{xcolor}
\usepackage{tikz}
\usetikzlibrary{arrows,automata}

\usepackage{fancyhdr}
\pagestyle{fancy}
% Head
\fancyhead[L]{}
\fancyhead[C]{\includegraphics[scale=0.4]{../logo.jpg}}
\fancyhead[R]{}
% Footer
\fancyfoot[C]{}
% Line
\renewcommand{\headrulewidth}{0pt}

%opening
\title{INFO-F408: Computability \& complexity}
\date{23 Octobre, 2017}
\author{Rémy Detobel}

\newtcolorbox{defBox}{%
colback=gray!5!white,colframe=black!75!white,fonttitle=\bfseries,
title=Définition}

\newtcolorbox{theoremeBox}{%
colback=blue!5!white,colframe=blue!40!white,fonttitle=\bfseries,
title=Théorème}

\newtcolorbox{importantBox}{%
colback=red!5!white,colframe=red!65!white,fonttitle=\bfseries,
title=Important}


\begin{document}

\maketitle
\newpage

\section{P vs NP}
  \begin{defBox}
    NP est un ensemble de langage vérifiable en temps polynomial.
  \end{defBox}
  \begin{theoremeBox}
    Un langage appartient à NP $\Leftrightarrow$ il peut être décidé dans une machine de turing non déterministe à temps polynomial.
  \end{theoremeBox}

  $$\text{SUBSET SUM} = \Big\{<S, t> | S = \{x_1, ..., x_k\} \text{ et pour certain } Y \subseteq \{1, ... k\}, \sum\limits_{j \in Y} x_j = t \Big\}$$
  
  \subsection{Problème du CLIQUE}
    \begin{tikzpicture}
      \node[draw, circle] (0) at (0,0) {};
      \node[draw, circle] (1) at (0,1) {};
      \node[draw, circle] (2) at (0,2) {};
      \node[draw, circle] (3) at (1,0.5) {};
      \node[draw, circle] (4) at (1,2) {};
      \node[draw, circle] (5) at (2,1) {};
      
      \draw (0) -- (1);
      \draw (0) -- (3);
      \draw (1) -- (2);
      \draw (1) -- (3);
      \draw (4) -- (3);
      \draw (1) -- (5);
      \draw (4) -- (5);
      \draw (3) -- (5);
      \draw (2) -- (4);
    \end{tikzpicture}\\
    Le problème CLIQUE peut dont être définit comme étant $= \big\{<G, K> |$ G est un graphe non dirigé avec k-clique$\big\}$\\
    
    \begin{minipage}[c]{0.49\linewidth}
      \begin{center}
	Exemple valide:\\
	\begin{tikzpicture}
	  \node[draw, circle, fill=green!20!gray] (0) at (0,0) {};
	  \node[draw, circle, fill=green!20!gray] (1) at (0,1) {};
	  \node[draw, circle] (2) at (0,2) {};
	  \node[draw, circle, fill=green!20!gray] (3) at (1,0.5) {};
	  \node[draw, circle] (4) at (1,2) {};
	  \node[draw, circle] (5) at (2,1) {};
	  
	  \draw (0) -- (1);
	  \draw (0) -- (3);
	  \draw (1) -- (2);
	  \draw (1) -- (3);
	  \draw (4) -- (3);
	  \draw (1) -- (5);
	  \draw (4) -- (5);
	  \draw (3) -- (5);
	  \draw (2) -- (4);
	\end{tikzpicture}
      \end{center}
    \end{minipage}
    \begin{minipage}[c]{.49\linewidth}
      \begin{center}
	Exemple d'ensemble ne répondant pas au problème:\\
	\begin{tikzpicture}
	  \node[draw, circle] (0) at (0,0) {};
	  \node[draw, circle] (1) at (0,1) {};
	  \node[draw, circle, fill=red] (2) at (0,2) {};
	  \node[draw, circle, fill=red] (3) at (1,0.5) {};
	  \node[draw, circle, fill=red] (4) at (1,2) {};
	  \node[draw, circle] (5) at (2,1) {};
	  
	  \draw (0) -- (1);
	  \draw (0) -- (3);
	  \draw (1) -- (2);
	  \draw (1) -- (3);
	  \draw (4) -- (3);
	  \draw (1) -- (5);
	  \draw (4) -- (5);
	  \draw (3) -- (5);
	  \draw (2) -- (4);
	\end{tikzpicture}
      \end{center}
    \end{minipage}
    \\
    \begin{defBox}
     Une fonction $f: \Sigma^{*} \rightarrow \Sigma^{*}$ est calculable dans un temps polynomial s'il existe une machine de turing tel qu'elle s'arrête avec seulement $f(w)$ sur son ruban avec une entrée w et s'exécute en un temps polynomial.
    \end{defBox}
    
    \begin{defBox}
      Un langage A est \underline{réductible en temps polynomial} dans un langage B, écrivons:
      $$A \leq_p B$$
      S'il existe une fonction $f$ s'exécutant en un temps polynimal tel que:
      $$\forall w, w \in A \Leftrightarrow f(w) \in B$$
    \end{defBox}

    \begin{importantBox}
      Si $A \leq_p B$ et $B \in P$, alors $A \in P$
    \end{importantBox}
    
    \subsubsection{Exemple: Problème de 3 SAT}
      \begin{align*}
	\phi = &(x_1 \vee x_1 \vee x_2) \wedge \\
	&(\overline{x_1} \vee \overline{x_2} \vee \overline{x_2}) \wedge \\
	&(\overline{x_1} \vee x_2 \vee x_2)
      \end{align*}
      Tel que
      \begin{align*}
	x_1 \leftarrow F\\
	x_2 \leftarrow T
      \end{align*}
      3-CNF formule (il s'agit d'une forme normal conjonctive)
      $$3\ SAT = \big\{\phi | \phi \text{ est une formule boolean satisfaisant 3-CNF } \big\}$$
      $3\ SAT \in NP$
    
      \begin{theoremeBox}
	$3\ SAT \leq_p$ CLIQUE
      \end{theoremeBox}
    
    \subsubsection{Passage de 3-SAT à CLIQUE}
      On va donc devoir trouver un algorithme pouvant transformer un problème ``3 SAT'' en un problème de type ``K-CLIQUE'' en un temps polynomial.  Cela permettra donc de résoudre ``3 SAT'' avec seulement un algorithme de résolution de type ``K-CLIQUE''.\\
      Un ``litéral'' est simplement une variable ou son inverse (ex: $x_i$ ou $\overline{x_i}$).
      
      \begin{center}
	\begin{tikzpicture}
	  \node[draw, circle] (1x1) at (0,4) {$x_1$};
	  \node[draw, circle] (1x12) at (0,2) {$x_1$};
	  \node[draw, circle] (1x2) at (0,0) {$x_2$};
	  
	  \node[draw, circle] (2nx1) at (2,6) {$\overline{x_1}$};
	  \node[draw, circle] (2nx2) at (4,6) {$\overline{x_2}$};
	  \node[draw, circle] (2nx22) at (6,6) {$\overline{x_2}$};
	  
	  \node[draw, circle] (3nx1) at (8,4) {$\overline{x_1}$};
	  \node[draw, circle] (3x2) at (8,2) {$x_2$};
	  \node[draw, circle] (3x22) at (8,0) {$x_2$};
	  
	  \draw (1x1) -- (2nx2);
	  \draw (1x1) -- (2nx22);
	  \draw (1x1) -- (3x2);
	  \draw (1x1) -- (3x22);
	  
	  \draw (1x12) -- (2nx2);
	  \draw (1x12) -- (2nx22);
	  \draw (1x12) -- (3x2);
	  \draw (1x12) -- (3x22);
	  
	  \draw (1x2) -- (2nx1);
	  \draw (1x2) -- (3nx1);
	  \draw (1x2) -- (3x2);
	  \draw (1x2) -- (3x22);
	  
	  \draw (2nx1) -- (3x2);
	  \draw (2nx1) -- (3x22);
	  \draw (2nx1) -- (3nx1);
	  
	  \draw (2nx2) -- (3nx1);
	  
	  \draw (2nx22) -- (3nx1);
	\end{tikzpicture}\\
      \end{center}
      $K = 3$, il s'agit donc du nombre de clause.\\
      On peut donc écrire que si $\phi$ est satisfait, G as un k-clique.\\
      \textbf{Une solution pourrait être:}\\
      \begin{minipage}{0.7\textwidth}
	\begin{center}
	  \begin{tikzpicture}
	    \node[draw, circle] (1x1) at (0,4) {$x_1$};
	    \node[draw, circle] (1x12) at (0,2) {$x_1$};
	    \node[draw, circle, white, fill=green!20!gray] (1x2) at (0,0) {$x_2$};
	    
	    \node[draw, circle, white, fill=green!20!gray] (2nx1) at (2,6) {$\overline{x_1}$};
	    \node[draw, circle] (2nx2) at (4,6) {$\overline{x_2}$};
	    \node[draw, circle] (2nx22) at (6,6) {$\overline{x_2}$};
	    
	    \node[draw, circle, white, fill=green!20!gray] (3nx1) at (8,4) {$\overline{x_1}$};
	    \node[draw, circle] (3x2) at (8,2) {$x_2$};
	    \node[draw, circle] (3x22) at (8,0) {$x_2$};
	    
	    \draw (1x1) -- (2nx2);
	    \draw (1x1) -- (2nx22);
	    \draw (1x1) -- (3x2);
	    \draw (1x1) -- (3x22);
	    
	    \draw (1x12) -- (2nx2);
	    \draw (1x12) -- (2nx22);
	    \draw (1x12) -- (3x2);
	    \draw (1x12) -- (3x22);
	    
	    \draw[color=green!20!gray,ultra thick] (1x2) -- (2nx1);
	    \draw[color=green!20!gray,ultra thick] (1x2) -- (3nx1);
	    \draw (1x2) -- (3x2);
	    \draw (1x2) -- (3x22);
	    
	    \draw (2nx1) -- (3x2);
	    \draw (2nx1) -- (3x22);
	    \draw[color=green!20!gray,ultra thick] (2nx1) -- (3nx1);
	    
	    \draw (2nx2) -- (3nx1);
	    
	    \draw (2nx22) -- (3nx1);
	  \end{tikzpicture}\\
	\end{center}
      \end{minipage}
      \begin{minipage}{0.29\textwidth}
	\begin{align*}
	  \phi = &(x_1 \vee x_1 \vee \mathbf{x_2}) \wedge \\
	  &(\mathbf{\overline{x_1}} \vee \overline{x_2} \vee \overline{x_2}) \wedge \\
	  &(\mathbf{\overline{x_1}} \vee x_2 \vee x_2)
	\end{align*}
      \end{minipage}
      
      Si l'on prend un littéral vrai sur chacune des lignes, on aura une solution au k-clique.
      
\section{NP-Compleet}
  \begin{defBox}
    Un langage B est ``NP-complet'' si et seulement si:
    \begin{enumerate}
      \item $B \in NP$
      \item Pour chaque $A \in NP, A \leq_p B$
    \end{enumerate}
  \end{defBox}
  
  \subsection{Cook-Levin Théorème}
    Définissons SAT comme étant un problème retournant un boolean pour toutes les formules sous forme normal conjonctive.\\
    SAT est NP-Complet.
    
    \begin{enumerate}
      \item Si B est NP-complet et $B \in P$, alors $P = NP$
      \item Si B est NP-complet et $B \leq_p C$ (pour $C \in NP$), alors C est NP-Complet.
    \end{enumerate}
    On peut donc écrire par exemple:
    \begin{align*}
      \text{Si } &A \leq_p B\\
      \wedge &B \leq_p C\\
      \Rightarrow &A \leq_p C
    \end{align*}

    
    \begin{defBox}
      Un langage B est NP-Hard (NP-Dur ou NP-Difficile) si et seulement si:\\
      pour tout $A \in NP, A \leq_p B$
    \end{defBox}
    Considérons $A \in NP$.  Il existe donc une machine de turing ``N'' non déterministe qui peut décider du langage A en un temps polynomial.  Supposons en temps $\leq n^k$ pour un certain $k = O(1)$
    
    \begin{align*}
      &\# q_0 w_1 w_2 ... w_n\ \_\ \_\ ... \_\ \_\#\\
      &\# a q_1 w_2 ... w_n\ \_\ \_\ ... \_\ \_\#\\
      &...
    \end{align*}
    Où on peut compter $n^k$ lignes sur une longueur de $n^k$\\
    $$C = Q \cup \Gamma \cup \{\#\}$$
    Variables: $X_{i, j, s} | i,j = 1...n^k, S \in C $
    
    $$\phi = \phi_{cell} \wedge \phi_{start} \wedge \phi_{move} \wedge \phi_{accept}$$
    
    \begin{align*}
      \phi_{start} =& X_{1, 1, \#} \wedge X_{1, 2, q_0} \wedge X_{1, 3, w_1} \wedge ... \wedge X_{1, n+2, w_n} \wedge X_{1, n+3, \_} \wedge ... \wedge X_{1, n^k-1, \_} \wedge X_{1, n^k, \#}\\
      \phi_{cell} =& \bigwedge\limits_{1 \leq i, j \leq n^k} \Big[ \big(\bigvee\limits_{S \in C} X_{i, j, s} \big) \wedge \bigwedge\limits_{s, t \in C | s \neq t} \big( \overline{X_{i, j, s}} \vee \overline{X_{i, j, t}} \big) \Big]\\
      \phi_{accept} =& \bigvee\limits_{1 \leq i, j \leq n^k} X_{i, j, q_{accept}}\\
      \mathbf{\phi_{move}} =& \bigwedge\limits_{i,j} \Big[\text{où }(i,j) \text{ windows est légal} \Big]\\
      \uparrow &\text{ ``Use Legal Windows'', il s'agit d'une fenêtre de 2 (ligne) par 3 (colonne)}\\
      &\text{ placé partout dans le tableau}
    \end{align*}
    Où une fenêtre légal est définit par:\\
    $\bigvee\limits_{a_1, a_2, ... a_6 \text{ est une fenêtre légal}} \big(X_{i, j-1, a_1} \wedge X_{i, j, a_2} \wedge X_{i, j+1, a_3} \wedge ... \big)$
    
    \begin{table}[h]
      \centering
      \begin{tabular}{cccc}
			      & j-1                       & j                         & j+1                       \\ \cline{2-4} 
      \multicolumn{1}{c|}{i}   & \multicolumn{1}{c|}{$a_1$} & \multicolumn{1}{c|}{$a_2$} & \multicolumn{1}{c|}{$a_3$} \\ \cline{2-4} 
      \multicolumn{1}{c|}{i+1} & \multicolumn{1}{c|}{$a_4$} & \multicolumn{1}{c|}{$a_5$} & \multicolumn{1}{c|}{$a_6$} \\ \cline{2-4} 
      \end{tabular}
    \end{table}
    
    \begin{table}[h]
      \centering
      \begin{tabular}{|c|c|c|}
      \hline
      \# & $q_0$ & $w_0$ \\ \hline
      \# & a    & $q_1$ \\ \hline
      \end{tabular}
    \end{table}
    est légal.
    
    $(q_1, a, R) \in \delta(q_0, w_1)$
    
    \begin{table}[h]
      \centering
      \begin{tabular}{|c|c|c|}
      \hline
      a & b & c \\ \hline
      a & d & e \\ \hline
      \end{tabular}
    \end{table}
    $d \neq b$ et $e \neq c$ et $\in \Gamma$ n'est pas légal
    
  
\end{document}